
%!TEX program = xelatex

\documentclass[UTF8]{ctexart}
\usepackage{ctex}

\CTEXsetup[format={\Large\bfseries}]{section}

\usepackage[version=3]{mhchem} % Package for chemical equation typesetting
\usepackage{siunitx} % Provides the \SI{}{} and \si{} command for typesetting SI units
\usepackage{graphicx} % Required for the inclusion of images
\usepackage{natbib} % Required to change bibliography style to APA
\usepackage{amsmath} % Required for some math elements 
\usepackage[hidelinks]{hyperref}
\usepackage{makecell} % 3 Packages for flexible tabular
\usepackage{multirow}
\usepackage{multicol}

\usepackage{geometry}% 版面大小
\geometry{a4paper,scale=0.7}

\usepackage{fontspec}

\setCJKfamilyfont{hwxk}{STXingkai}% 字体
\newcommand{\hwxk}{\CJKfamily{hwxk}}

\usepackage{fancyhdr}% 页眉页脚
\fancypagestyle{EE_template}{
    \fancyhead[L]{\Large {\hwxk 南京大学电子科学与工程学院}}    
    \fancyhead[R]{模拟电路实验报告}
    \fancyfoot[c]{- \thepage \ -}
    \renewcommand\footrulewidth{0pt}
}

% 4级目录
\setcounter{secnumdepth}{4}
\setcounter{tocdepth}{4}

\usepackage{graphicx} % Packages for figures
\usepackage{caption2}
\usepackage{subfigure}
\usepackage{float}


%设置图片、表格编号
\renewcommand{\thetable}{\thesubsection{}-\arabic{table}}
\renewcommand{\thefigure}{\thesubsection{}-\arabic{figure}}
\renewcommand{\thefigure}{\thesubsection{}-\arabic{equation}}
\usepackage{amsmath}
\numberwithin{figure}{subsection}
\numberwithin{table}{subsection}
\numberwithin{equation}{subsection}

\setlength\parindent{6pt} % Removes all indentation from paragraphs

\renewcommand{\labelenumi}{\alph{enumi}.} % Make numbering in the enumerate environment by letter rather than number (e.g. section 6)

%\usepackage{times} % Uncomment to use the Times New Roman font

%----------------------------------------------------------------------------------------
%	DOCUMENT INFORMATION
%----------------------------------------------------------------------------------------

\title{\textbf{实验4\ }} % Title

\author{电子科学与工程学院 201180078 刘时宜} % Author name

\date{} % Date for the report

\begin{document}

\pagestyle{EE_template}

\maketitle % Insert the title, author and date

\begin{center}
\begin{tabular}{l r}
实验日期: & 2021年5月20日 \\ % Date the experiment was performed
指导老师: & 郑江 % Instructor/supervisor
\end{tabular}
\end{center}

% If you wish to include an abstract, uncomment the lines below
% \begin{abstract}
% Abstract text
% \end{abstract}


\section{正弦波发生器}

\begin{table}[h]
    \begin{center}
        \caption{测量正弦波发生器频率随\(R_4+R_P\)的变化}
        \begin{tabular}{|c|c|c|}
            \hline
            \(R_4+R_P\cdot \SI{}{\per\kilo\ohm}\) & \(V_{out}\)峰峰值 & 周期\(T\) \\
            \hline
            19.970 & \SI{11.375}{\volt} & \SI{3.30}{\milli\second} \\
            \hline
            40.069 & \SI{11.375}{\volt} & \SI{5.42}{\milli\second} \\
            \hline
            60.034 & \SI{11.375}{\volt} & \SI{7.18}{\milli\second} \\
            \hline
            79.974 & \SI{11.375}{\volt} & \SI{8.88}{\milli\second} \\
            \hline
            100.017 & \SI{11.375}{\volt} & \SI{10.45}{\milli\second} \\
            \hline
        \end{tabular}
    \end{center}
    \label{step wave f exp data}
\end{table}

\section{占空比可调的矩形波发生器}
\begin{table}[h]
    \begin{center}
        \caption{占空比可调的矩形波发生器(实验)}
        \begin{tabular}{|c|c|c|c|c|c|c|c|}
            \hline
            \multicolumn{5}{|c|}{实验测量} & \multicolumn{3}{c|}{理论值} \\
            \hline
            \(R_{PP}\) & \(V_o\)峰峰值 & \(T\cdot\SI{}{\per\milli\second}\) & \(t_H\cdot\SI{}{\per\milli\second}\) & 占空比 & \(T\) & \(t_H\) & 占空比 \\
            \hline
            \SI{10.095}{\kilo\ohm} & \SI{11.375}{\volt} & 7.60 & 1.80 & 23.68\% & & & \\
            \hline
            \SI{30.074}{\kilo\ohm} & \SI{11.375}{\volt} & 7.98 & 3.00 & 37.59\% & & & \\
            \hline
            \SI{50.019}{\kilo\ohm} & \SI{11.375}{\volt} & 7.98 & 3.96 & 49.62\% & & & \\
            \hline
            \SI{70.062}{\kilo\ohm} & \SI{11.375}{\volt} & 7.96 & 4.90 & 61.55\% & & & \\
            \hline
            \SI{90.013}{\kilo\ohm} & \SI{11.375}{\volt} & 7.68 & 5.78 & 75.26\% & & & \\
            \hline
        \end{tabular}
        \par \(R_P = \SI{100.622}{\kilo\ohm}\)
    \end{center}
    \label{PWM exp data}
\end{table}

\section{三角波发生器}
\begin{table}[h]
    \begin{center}
        \caption{\(R_P\)对三角波周期的影响(实验)}
        \begin{tabular}{|c|c|c|c|c|c|}
            \hline
            \multicolumn{4}{|c|}{测量值} & \multicolumn{2}{c|}{理论值} \\
            \hline
            \(C\) & \(R_P\) & \(V_o\)峰峰值 & \(T\cdot\SI{}{\per\milli\second}\) & \(V_o\)峰峰值 & \(T\) \\
            \hline
            \SI{0.22}{\micro\farad} & \SI{5.049}{\kilo\ohm} & \SI{5.400}{\volt} & 4.88 & & \\
            \hline
            \SI{0.22}{\micro\farad} & \SI{9.994}{\kilo\ohm} & \SI{10.625}{\volt} & 9.44 & & \\
            \hline
            \SI{0.22}{\micro\farad} & \SI{15.012}{\kilo\ohm} & \SI{15.750}{\volt} & 14.00 & & \\
            \hline
            \SI{0.22}{\micro\farad} & \SI{18.916}{\kilo\ohm} & \SI{19.6875}{\volt} & 17.60 & & \\
            \hline
        \end{tabular}
    \end{center}
    \label{triangular wave f exp data}
\end{table}


\section{锯齿波发生器}
\begin{table}[h]
    \begin{center}
        \caption{\(R_PP\)对锯齿波的影响(实验)}
        \begin{tabular}{|c|c|c|c|}
            \hline
            \multicolumn{4}{|c|}{实验测量值} \\
            \hline
            \(R_{PP}\)  & 上升时间\(T_1\) & 下降时间\(T_2\) & 周期\(T\)\\
            \hline
            \SI{10.064}{\kilo\ohm} & \SI{10.6}{\milli\second} & \SI{51.4}{\milli\second} & \SI{61.8}{\milli\second} \\
            \hline
            \SI{49.62}{\kilo\ohm} & \SI{31.2}{\milli\second} & \SI{31.2}{\milli\second} & \SI{62.8}{\milli\second} \\
            \hline
            \SI{90.059}{\kilo\ohm} & \SI{52.0}{\milli\second} & \SI{10.8}{\milli\second} & \SI{62.8}{\milli\second} \\
            \hline
            
        \end{tabular}
        \par \(R_P = \SI{100.594}{\kilo\ohm}\)
    \end{center}
    \label{sawtooth wave f exp data}
\end{table}




\end{document}