%!TEX program = xelatex

\documentclass[UTF8]{ctexart}
\usepackage{ctex}

\CTEXsetup[format={\Large\bfseries}]{section}

\usepackage[version=3]{mhchem} % Package for chemical equation typesetting
\usepackage{siunitx} % Provides the \SI{}{} and \si{} command for typesetting SI units
\usepackage{graphicx} % Required for the inclusion of images
\graphicspath{{assets/}}
\usepackage{natbib} % Required to change bibliography style to APA
\usepackage{amsmath} % Required for some math elements 
\usepackage{amssymb}
\usepackage[hidelinks]{hyperref}
\usepackage{makecell} % 3 Packages for flexible tabular
\usepackage{multirow}
\usepackage{multicol}

\usepackage{pdfpages}  % include pdf pages for original data etc.

\usepackage{geometry}% 版面大小
\geometry{a4paper,scale=0.7}

\usepackage{fontspec}

\setCJKfamilyfont{hwxk}{STXingkai}% 字体
\newcommand{\hwxk}{\CJKfamily{hwxk}}

\usepackage{fancyhdr}% 页眉页脚
\fancypagestyle{EE_Digital1Exp_template}{
    \fancyhead[L]{\Large {\hwxk 南京大学电子科学与工程学院}}
    \fancyhead[R]{数字系统1实验报告}
    \fancyfoot[c]{- \thepage \ -}
    \renewcommand\footrulewidth{0pt}
}

% 4级目录
\setcounter{secnumdepth}{4}
\setcounter{tocdepth}{4}

\usepackage{graphicx} % Packages for figures
\usepackage{caption2}
\usepackage{subfigure}
\usepackage{float}


%设置图片、表格编号
\renewcommand{\thetable}{\thesubsection{}-\arabic{table}}
\renewcommand{\thefigure}{\thesubsection{}-\arabic{figure}}
\renewcommand{\thefigure}{\thesubsection{}-\arabic{equation}}
\usepackage{amsmath}
\numberwithin{figure}{subsection}
\numberwithin{table}{subsection}
\numberwithin{equation}{subsection}

\setlength\parindent{6pt} % Removes all indentation from paragraphs

\renewcommand{\labelenumi}{\alph{enumi}.} % Make numbering in the enumerate environment by letter rather than number (e.g. section 6)

%\usepackage{times} % Uncomment to use the Times New Roman font

%----------------------------------------------------------------------------------------
%	DOCUMENT INFORMATION
%----------------------------------------------------------------------------------------

\title{\textbf{实验2\ 计数器}} % Title

\author{电子科学与工程学院\ 刘时宜\ 201180078} % Author name

\date{} % Date for the report

\begin{document}

\pagestyle{EE_Digital1Exp_template}

\maketitle % Insert the title, author and date

\begin{center}
    \begin{tabular}{l r}
    实验日期: & 2021年11月3日 \\ % Date the experiment was performed
     & 2021年11月10日 \\ % Date the experiment was performed
    指导老师: & 高健 % Instructor/supervisor
    \end{tabular}
    \par 点击目录、书签栏、以及行文中的图表标号的均可跳转至相应页面
    \end{center}
    
% If you wish to include an abstract, uncomment the lines below
% \begin{abstract}
% Abstract text
% \end{abstract}

\tableofcontents

\section{实验目的}
\begin{enumerate}
    \item 验证计数器工作原理
    \item 利用已有计数器模块设计实现其他进制的计数器模块
    \item 利用计数器设计实现秒表
\end{enumerate}

\section{实验仪器与主要器材}
\begin{center}
    \begin{tabular}{ll}
        \textbf{仪器:} & \\
        Basys3 FPGA 开发板 & 1台\\
        KEYSIGHT DSOX1102AG 示波器 & 1台\\
        示波器高频探头 & 1套\\
        ROGOL DM3068 万用表 & 1台\\
        \textbf{软件:} & \\
        Multisim & 14.1 \\
        Digilent Adept & 2.19.2 \\
        Vivado & 2015.4 \\
        \textbf{耗材:} & \\
        导线 & 若干 \\
    \end{tabular}
\end{center}

\section{实验原理}
\par 计数器在数字系统中广泛使用,可以用于对时钟脉冲计数、分频、定时、进行数学运算等。
\par 计数器种类繁多。如果按照计数器中的触发器是否同时翻转分类,可以将计数器分为同步式和异步式两种。同步计数器中,当时钟脉冲输入时触发器的翻转同时发生,而在异步计数器中,触发器的翻转又先后,不是同时发生的。
\par 如果按照计数过程中数字的增减分类,可以分为加法计数器、减法计数器以及可逆计数器。随着计数脉冲的不断输入而作递增的计数器称为加法计数器,作递减计数的称为减法计数器,可增可减的称为可逆计数器。
\par 本实验中用到的计数器多为异步加法计数器。

\section{实验过程}
\subsection{十进制加法计数器}
开发板连接至电脑后下载已经编译好的十进制计数器bit流文件。
\par 将\(SW_0\)、\(SW_1\)开关均拨至断开位置,观察计数器数字缓慢变化。将\(SW_1\)闭合,使用示波器观察计数器各输出引脚波形。
\subsubsection{实验结果}
\(SW_0 = SW_1 = 0\)时,时钟周期为\SI{0.75}{\Hz},计数器数字缓慢变化。观察到十进制计数器共有10个稳态,分别显示数字\(0\sim 9\),如图\ref{10counter_slow}所示。

\begin{figure}[H]
    \centering
    \subfigure[0]{
    \includegraphics[width=0.3\textwidth]{original pic/0.jpg}}
    \subfigure[1]{
    \includegraphics[width=0.3\textwidth]{original pic/1.jpg}}
    \subfigure[2]{
    \includegraphics[width=0.3\textwidth]{original pic/2.jpg}}
    \subfigure[3]{
    \includegraphics[width=0.3\textwidth]{original pic/3.jpg}}
    \subfigure[4]{
    \includegraphics[width=0.3\textwidth]{original pic/4.jpg}}
    \subfigure[5]{
    \includegraphics[width=0.3\textwidth]{original pic/5.jpg}}
    \subfigure[6]{
    \includegraphics[width=0.3\textwidth]{original pic/6.jpg}}
    \subfigure[7]{
    \includegraphics[width=0.3\textwidth]{original pic/7.jpg}}
    \subfigure[8]{
    \includegraphics[width=0.3\textwidth]{original pic/8.jpg}}
    \subfigure[9]{
    \includegraphics[width=0.3\textwidth]{original pic/9.jpg}}
    
    \caption{十进制计数器数码管显示}
    \label{10counter_slow}
\end{figure}

\(SW_0 =0 ,  SW_1 = 1\)时,时钟周期为\SI{3}{\kHz},数码管由于视觉暂留现象全亮,此时使用示波器观察计数器各输出引脚波形如图\ref{10counter_fast}所示。可以看出,在计数器一个全周期内共经历了10个时钟周期,与十进制计数器要求相符。

\begin{figure}[H]
    \centering
    \subfigure[\(JB_1\)]{
    \includegraphics[width=0.45\textwidth]{10/jb1.jpg}}
    \subfigure[\(JB_2\)]{
    \includegraphics[width=0.45\textwidth]{10/jb2.jpg}}
    \subfigure[\(JB_3\)]{
    \includegraphics[width=0.45\textwidth]{10/jb3.jpg}}
    \subfigure[\(JB_4\)]{
    \includegraphics[width=0.45\textwidth]{10/jb4.jpg}}

    
    \caption{十进制计数器引脚输出波形}
    \label{10counter_fast}
\end{figure}

\subsubsection{小结}
\begin{itemize}
    \item 观察了十进制计数器连接数码管可以实现\(0 \sim 9\)数字循环显示的功能。
    \item 测量了十进制计数器的输出波形,符合10个时钟周期1循环的要求。
\end{itemize}

\subsection{16进制加法计数器}
开发板连接至电脑后下载已经编译好的16进制计数器bit流文件。
\par 将\(SW_0\)拨至闭合位置、\(SW_1\)拨至断开位置,观察计数器数字缓慢变化。将\(SW_1\)闭合,使用示波器观察计数器各输出引脚波形。
\subsubsection{实验结果}
\(SW_0 =1, SW_1 = 0\)时,时钟周期为\SI{0.75}{\Hz},计数器数字缓慢变化。观察到十进制计数器共有10个稳态,分别显示数字\(0\sim 9\),以及\(10\sim 15\)所对应输出数码,如图\ref{16counter_slow}所示。

\begin{figure}[H]
    \centering
    \subfigure[0]{
    \includegraphics[width=0.22\textwidth]{original pic/0.jpg}}
    \subfigure[1]{
    \includegraphics[width=0.22\textwidth]{original pic/1.jpg}}
    \subfigure[2]{
    \includegraphics[width=0.22\textwidth]{original pic/2.jpg}}
    \subfigure[3]{
    \includegraphics[width=0.22\textwidth]{original pic/3.jpg}}
    \subfigure[4]{
    \includegraphics[width=0.22\textwidth]{original pic/4.jpg}}
    \subfigure[5]{
    \includegraphics[width=0.22\textwidth]{original pic/5.jpg}}
    \subfigure[6]{
    \includegraphics[width=0.22\textwidth]{original pic/6.jpg}}
    \subfigure[7]{
    \includegraphics[width=0.22\textwidth]{original pic/7.jpg}}
    \subfigure[8]{
    \includegraphics[width=0.22\textwidth]{original pic/8.jpg}}
    \subfigure[9]{
    \includegraphics[width=0.22\textwidth]{original pic/9.jpg}}
    \subfigure[10]{
    \includegraphics[width=0.22\textwidth]{original pic/10.jpg}}
    \subfigure[11]{
    \includegraphics[width=0.22\textwidth]{original pic/11.jpg}}
    \subfigure[12]{
    \includegraphics[width=0.22\textwidth]{original pic/12.jpg}}
    \subfigure[13]{
    \includegraphics[width=0.22\textwidth]{original pic/13.jpg}}
    \subfigure[14]{
    \includegraphics[width=0.22\textwidth]{original pic/14.jpg}}
    \subfigure[15]{
    \includegraphics[width=0.22\textwidth]{original pic/15.jpg}}

    
    \caption{16进制计数器数码管显示}
    \label{16counter_slow}
\end{figure}

\(SW_0 =1 ,  SW_1 = 1\)时,时钟周期为\SI{3}{\kHz},数码管由于视觉暂留现象全亮,此时使用示波器观察计数器各输出引脚波形如图\ref{16counter_fast}所示。可以看出,在计数器一个全周期内共经历了16个时钟周期,与16进制计数器要求相符。

\begin{figure}[H]
    \centering
    \subfigure[\(JB_1\)]{
    \includegraphics[width=0.45\textwidth]{16/jb1.jpg}}
    \subfigure[\(JB_2\)]{
    \includegraphics[width=0.45\textwidth]{16/jb2.jpg}}
    \subfigure[\(JB_3\)]{
    \includegraphics[width=0.45\textwidth]{16/jb3.jpg}}
    \subfigure[\(JB_4\)]{
    \includegraphics[width=0.45\textwidth]{16/jb4.jpg}}

    
    \caption{16进制计数器引脚输出波形}
    \label{16counter_fast}
\end{figure}

\subsubsection{小结}
\begin{itemize}
    \item 观察了16进制计数器连接数码管可以实现16种数码循环显示的功能。
    \item 测量了16进制计数器的输出波形,符合16个时钟周期1循环的要求。
\end{itemize}

\subsection{六进制计数器(置零法)}

\subsubsection{实验步骤}
\par 利用计数器的置零端,使得计数器输出为二进制6,即“0110”时,计数器置零端全部为高电平,将计数器输出置零,使得计数器仅有数字0到数字5共6个稳态,成为六进制计数器。
\par 改变置零端连接后的电路图如图\ref{6set0 circuit}所示。
\par 生成bit文件,下载到开发板上,观察实验现象。

\begin{figure}[H]
    \begin{center}
        \includegraphics[width=0.8\textwidth]{6set0/ciruit.png}
    \end{center}
    \caption{六进制计数器(置零法)电路图}
    \label{6set0 circuit}
\end{figure}


\subsubsection{实验结果}
\(SW_0 = SW_1 = 0\)时,时钟周期为\SI{0.75}{\Hz},计数器数字缓慢变化。观察到十进制计数器共有6个稳态,分别显示数字\(0\sim5\),实验现象已录制成为视频“6进制计数器(置零法).mp4”,附在邮件中。

\par \(SW_0 =0 ,  SW_1 = 1\)时,时钟周期为\SI{3}{\kHz},数码管由于视觉暂留现象全亮,此时使用示波器观察计数器各输出引脚波形如图\ref{6set0}所示。可以看出,在计数器一个全周期内共经历了6个时钟周期,与六进制计数器要求相符。

\begin{figure}[H]
    \centering
    \subfigure[\(JB_1\)]{
    \includegraphics[width=0.45\textwidth]{6set0/jb1.jpg}}
    \subfigure[\(JB_2\)]{
    \includegraphics[width=0.45\textwidth]{6set0/jb2.jpg}}
    \subfigure[\(JB_3\)]{
    \includegraphics[width=0.45\textwidth]{6set0/jb3.jpg}}
    \subfigure[\(JB_4\)]{
    \includegraphics[width=0.45\textwidth]{6set0/jb4.jpg}}

    \caption{六进制计数器引脚输出波形}
    \label{6set0}
\end{figure}

\subsubsection{小结}
\begin{itemize}
    \item 熟悉了计数器置零端的用法,掌握了利用置零端使计数器数字归零,调整进位的方法。
    \item 利用置零法,在输出数字6的时候将数字置零,实现了六进制计数器。
\end{itemize}

\subsection{六进制计数器(置9法)}

\subsubsection{实验步骤}
\par 将原有计数器模块改为带置数端的计数器模块,将二进制数字9,即“1001”连接到计数器置数端。利用与非门,使得当且仅当输出为二进制4时置数使能端(\(\sim\)load)为低电平,当且仅当输出为二进制10时置零端为低电平。将计数器两使能端连接高电平,使计数器始终工作。更改后的电路图如图\ref{6set9 circuit}所示。
\par 生成bit文件,下载到开发板上,观察实验现象。

\begin{figure}[H]
    \begin{center}
        \includegraphics[width=0.8\textwidth]{6set9/circuit.png}
    \end{center}
    \caption{六进制计数器(置零法)电路图}
    \label{6set9 circuit}
\end{figure}


\subsubsection{实验结果}
\(SW_0 = SW_1 = 0\)时,时钟周期为\SI{0.75}{\Hz},计数器数字缓慢变化。观察到十进制计数器共有6个稳态,分别显示数字\(0\sim4\)以及数字9,实验现象已录制成为视频“6进制计数器(置9法).mp4”,附在邮件中。

\par \(SW_0 =0 ,  SW_1 = 1\)时,时钟周期为\SI{3}{\kHz},数码管由于视觉暂留现象全亮,此时使用示波器观察计数器各输出引脚波形如图\ref{6set9}所示。可以看出,在计数器一个全周期内共经历了6个时钟周期,与六进制计数器要求相符。

\begin{figure}[H]
    \centering
    \subfigure[\(JB_1\)]{
    \includegraphics[width=0.45\textwidth]{6set9/jb1.jpg}}
    \subfigure[\(JB_2\)]{
    \includegraphics[width=0.45\textwidth]{6set9/jb2.jpg}}
    \subfigure[\(JB_3\)]{
    \includegraphics[width=0.45\textwidth]{6set9/jb3.jpg}}
    \subfigure[\(JB_4\)]{
    \includegraphics[width=0.45\textwidth]{6set9/jb4.jpg}}

    \caption{六进制计数器引脚输出波形}
    \label{6set9}
\end{figure}

\subsubsection{小结}
\begin{itemize}
    \item 熟悉了计数器置数端的用法,掌握了利用置数端以及置零端使计数器数字归零,调整进位的方法。
    \item 利用置9法,实现了六进制计数器。
\end{itemize}

\subsection{秒表设计实现}
利用计数器设计一个秒表,最小精度为\SI{0.1}{\second}。每\SI{0.1}{\second},秒表跳一个数字,计10次,秒位加一,秒位计60次,分位加一,四位数码管总计时容量约10分钟。
\par {\color{red}\textbf{本秒表设计除实验材料中给定的秒分频电路、动态显示模块外,均为自主设计完成,未使用任何其他已有的的设计资料!}}

\subsubsection{秒表设计与功能实现}
\paragraph{按秒表所需进制数搭建计数器}~
\par 由设计要求,最低位为10进制计数器,第2、3位共同组成60进制计数器,最高位组成10进制计数器。计数器串联获得秒表所需的计数器。具体实现方式如下:
\begin{enumerate}
    \item 搭建最低位的10进制计数器,时钟信号CLK使用秒分频电路产生的\SI{0.1}{\second}周期信号。
    \item 搭建第2位10进制计数器,时钟信号为最低位计数器的清零信号(\(\sim\)CLR)取反。取反的原因是由于采用的计数器模块触发方式为上升沿。
    \item 搭建第3位6进制计数器,时钟信号为第2位计数器的清零信号(\(\sim\)CLR)取反。
    \item 为确保第2、3位实现60进制计数器时需使第3位输出6时同时将第2为置零。则将第2位计数器置数端输入“0000”,将第3位的清零信号连接第2位置数端。
    \item 搭建最高位10进制计数器,时钟信号为第3位计数器的清零信号(\(\sim\)CLR)取反。
    \item 将各位计数器输出连接至动态显示模块相应端口,实现对数码管的显示控制。
\end{enumerate}

\paragraph{对于计数状态的控制}~
\par 设计要求中,要求实现开始、暂停、继续、复位的功能,使用1个开关和1个按键,结合计数器控制端即可实现,具体解决方案入下:
\begin{enumerate}
    \item 利用计数器的使能端ENT为低电平时,计数暂停的性质,将各数位上计数器的ENT端均连接到开关SW0,则可以实现使用开关SW0达到开始、暂停、继续的功能。
    \item 利用计数器的置零端,将各计数器原有置零端信号与按键信号取反后所得的信号取“与”运算即可实现在不影响原有功能下,加入按键清零的功能。
\end{enumerate}

\par 以上两部分的电路图如图\ref{watch left}所示。搭建完成的电路即可完成计时以及控制计时开始、暂停、继续和复位的功能。

\begin{figure}[H]
    \begin{center}
        \includegraphics[width=0.8\textwidth]{watch/left half.png}
    \end{center}
    \caption{计数器电路模块}
    \label{watch left}
\end{figure}

\paragraph{暂停时数字闪烁(附加功能)}~
\par 希望能够扩展做到计时暂停时数字闪烁,具体分析其需求如下:
\begin{enumerate}
    \item 初始时,秒表稳定显示“0000”,不闪烁。
    \item 正常计时过程中,稳定显示当前计时数字,不闪烁。
    \item 当开始计时又暂停时,数字闪烁。
\end{enumerate}

\par 使用1s的时钟信号作为闪烁信号的闪烁来源,则可以画出功能表如下表所示:
\begin{table}[h]
    \begin{center}
        \begin{tabular}{c c c|c}
            秒表显示“0000” & 暂停 & 时钟信号 & 输出 \\
            \hline
            1 & x & x & 数字 \\
            0 & 0 & x & 数字 \\
            0 & 1 & 0 & 0 \\
            0 & 1 & 1 & 数字 \\
        \end{tabular}
        \caption{实现闪烁功能的功能表}
    \end{center}
\end{table}

\par 注意暂停时SW0为低电平,则可以使用如图\ref{spark logic circuit}所示的逻辑电路实现上表:
\begin{figure}[H]
    \begin{center}
        \includegraphics[width=0.8\textwidth]{watch/spark logic circuit.jpg}
    \end{center}
    \caption{闪烁逻辑功能实现}
    \label{spark logic circuit}
\end{figure}

\par 最后,将所得的闪烁控制信号与原信号的反取“与非”关系控制响应数码管即可实现闪烁控制的功能。之所以要取反和取与非的原因是所用数码管为共阳极数码管,低电平时点亮。这部分电路如图\ref{watch right}所示。

\begin{figure}[H]
    \begin{center}
        \includegraphics[width=0.8\textwidth]{watch/right half.png}
    \end{center}
    \caption{闪烁逻辑功能模块}
    \label{watch right}
\end{figure}

\par 到此即完成了秒表全部功能的实现,包括计时、开始、暂停、继续、清零、暂停时闪烁。完整电路图如图\ref{watch circuit}所示。

\begin{figure}[H]
    \begin{center}
        \includegraphics[width=0.8\textwidth]{watch/whole circuit.png}
    \end{center}
    \caption{秒表全电路图}
    \label{watch circuit}
\end{figure}

\subsubsection{实验验证}
\par 将电路设计下载到Basys3开发板上,验证了设计的电路完全满足实验要求。验证视频为邮件附件中“秒表实验.mp4”。

\subsubsection{小结}
\begin{enumerate}
    \item 利用计数器设计实现了秒表的计时功能。
    \item 设计实现了秒表的计时状态控制功能。
    \item 设计实现了秒表暂停时闪烁的功能。
\end{enumerate}

\section{实验总结}
\begin{enumerate}
    \item 了解了计数器的基本工作原理与输出波形
    \item 掌握了利用已有计数器模块实现不同数值计数器的方法
    \item 利用计数器设计实现了秒表
\end{enumerate}

\section*{原始数据}

\begin{figure}[H]
    \begin{center}
        \includegraphics[width=0.8\textwidth]{original pic/sign.png}
    \end{center}
\end{figure}


\includepdf[pages=1-7]{exp_data.pdf}

\end{document}